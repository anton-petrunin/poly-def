\documentclass[a4paper,10pt]{article}
\usepackage{paper-en}
\usepackage{hyperref}




%\usepackage[notref,notcite,color]{showkeys}


\def\thetitle{Polyhedral spaces}
\def\theauthors{}

\hypersetup{colorlinks=true,
citecolor=black,
linkcolor=black,
anchorcolor=black,
filecolor=black,
menucolor=black,
urlcolor=black,
pdftitle={\thetitle},
pdfauthor={\theauthors}
}








%\usepackage[a-2b,mathxmp]{pdfx}[2018/12/22]

%\overfullrule=100mm

\begin{document}
%\pagestyle{empty}\renewcommand\includegraphics[2][{}]{}


\title{\thetitle}
\author{\theauthors}
\date{}
\maketitle

\begin{abstract}

\end{abstract}

\section{Introduction}

This paper should discuss different definitions of polyhdral sapaces and their basic properties, including the natural stratification and maybe holonomy group; anything else?
(We should not get into curvature bounds, otherwise it will be infinite.)

Let us define polyhedral space as a simplisial complex $S$ equipped with length metric such that each simplex is isometric to a Eucliden simplex.
(There are spherical and hyperbolic analogs.)

One may put extra conditions on complex and metric:
\begin{itemize}
\item $S$ is locally finite; equivalently, the space is locally compact.
\item Pure: any simplex in $S$ is a face of some $n$-simplex for some fixed $n\ge 0$.
\item Normal: $S$ is normal and any curve in $S$ can be approximated by a curve that runs in the simplicies of codimension 0 and 1.
\item Pseudomanifold/triangulated manifold/PL-manifold:...
\item Completeness of metric.
\end{itemize}

Here is an incomplete list of texts where polyhedral spaces were defined: \cite{alexander-kapovitch-petrunin2019,alexander-kapovitch-petrunin2024,botero-gil-sombra,burago-burago-ivanov,lebedeva-petrunin,milka1968,milka1969,minemyer2015,bridson-haefliger,panov2009}; it does not include polyhedral surfaces.

\section{Conical domains}

Let $\Sigma$ be a metric space with diameter at most $\pi$.
Consider the topological cone
$K=[0,\infty)\times\Sigma/{\sim}$, where $(0,x)\sim(0,y)$ for all $x,y\in\Sigma$.
We equip $K$ with the metric defined by the rule of cosines;
that is, for any $a,b\in[0,\infty)$ and $x,y\in\Sigma$,
we set
$$|(a,x)-(b,y)|_K^2=a^2+b^2-2\cdot a\cdot b\cdot\cos |x-y|_\Sigma.$$
The resulting space $K$ is called the \emph{Euclidean cone over} $\Sigma$.
All pairs of the form $(0,x)$ correspond to a single point in $K$,
referred to as the \emph{tip} of the cone.
A metric space that can be obtained in this way is called a \emph{Euclidean cone}.


\begin{thm}{Definition}
Let $X$ be a metric space, $x\in X$, and let $U$ be a neighborhood of $x$.
We say that $U$ is a \emph{conic neighborhood} of $x$
if there exists an open distance-preserving embedding
$\iota \: U \to K_x$
into a Euclidean cone $K_x$
such that $\iota(x)$ is the tip of the cone.
\end{thm}

Recall that a geodesic is a distance-preserving map from a real interval into a metric space.

\begin{thm}{Claim}
An open convex set $U$ in a geodesic metric space is a conic neighborhood of $p\in U$ if and only if there is a function $f\:U\to\RR$ with minmum at $p$ satisfying the differenitial equation $f''=1$ in $U$;
that is, $(f\circ\gamma)''(t)\equiv 1$ for any geodesic $\gamma$ in $U$.
\end{thm}

Maybe we need generalize it to $p\notin U$.

\section{Tringulations}

We should prove that two triangulations of a polyhedral space are equivalent;
that is, thay have a common subdivision.
What else?

The following statement should help and it should also lead to a better version of the main theorem in \cite{lebedeva-petrunin}.

\begin{thm}{Theorem}
Suppose a compact metric space $K$ admits a finite covering by conic balls $B(v_1,r_1),\dots,B(v_k,r_k)$,
then the nerve $N$ of the covering admits a map to $K$ that sends each simplex to an isometric copy of Euclidean polytop in $K$ and each vetex of $N$ is mapped to the center of the corresponding ball.
\end{thm}

Maybe it is possible to make a triangulation of $K$ with these vertices?



{\sloppy
\printbibliography[heading=bibintoc]
\fussy
}

\end{document}
